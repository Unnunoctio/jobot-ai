\documentclass[10pt, letterpaper]{article}

% Packages:
\usepackage[
    ignoreheadfoot, % set margins without considering header and footer
    top=2 cm, % seperation between body and page edge from the top
    bottom=2 cm, % seperation between body and page edge from the bottom
    left=2 cm, % seperation between body and page edge from the left
    right=2 cm, % seperation between body and page edge from the right
    footskip=1.0 cm, % seperation between body and footer
    % showframe % for debugging 
]{geometry} % for adjusting page geometry
\usepackage{titlesec} % for customizing section titles
\usepackage{tabularx} % for making tables with fixed width columns
\usepackage{array} % tabularx requires this
\usepackage[dvipsnames]{xcolor} % for coloring text
\definecolor{primaryColor}{RGB}{0, 0, 0} % define primary color
\usepackage{enumitem} % for customizing lists
\usepackage{fontawesome5} % for using icons
\usepackage{amsmath} % for math
\usepackage[
    pdftitle={Rafael Carvacho's CV},
    pdfauthor={Rafael Carvacho},
    pdfcreator={LaTeX with RenderCV},
    colorlinks=true,
    urlcolor=primaryColor
]{hyperref} % for links, metadata and bookmarks
\usepackage[pscoord]{eso-pic} % for floating text on the page
\usepackage{calc} % for calculating lengths
\usepackage{bookmark} % for bookmarks
\usepackage{lastpage} % for getting the total number of pages
\usepackage{changepage} % for one column entries (adjustwidth environment)
\usepackage{paracol} % for two and three column entries
\usepackage{ifthen} % for conditional statements
\usepackage{needspace} % for avoiding page brake right after the section title
\usepackage{iftex} % check if engine is pdflatex, xetex or luatex
\usepackage{datatool}
\usepackage{etoolbox}
\usepackage{xstring}

% Ensure that generate pdf is machine readable/ATS parsable:
\ifPDFTeX
    \input{glyphtounicode}
    \pdfgentounicode=1
    \usepackage[T1]{fontenc}
    \usepackage[utf8]{inputenc}
    \usepackage{lmodern}
\fi

\usepackage{charter}

% Some settings:
\raggedright
\AtBeginEnvironment{adjustwidth}{\partopsep0pt} % remove space before adjustwidth environment
\pagestyle{empty} % no header or footer
\setcounter{secnumdepth}{0} % no section numbering
\setlength{\parindent}{0pt} % no indentation
\setlength{\topskip}{0pt} % no top skip
\setlength{\columnsep}{0.15cm} % set column seperation
\pagenumbering{gobble} % no page numbering

\titleformat{\section}{\needspace{4\baselineskip}\bfseries\large}{}{0pt}{}[\vspace{1pt}\titlerule]

\titlespacing{\section}{
    % left space:
    -1pt
}{
    % top space:
    0.3 cm
}{
    % bottom space:
    0.2 cm
} % section title spacing

\renewcommand\labelitemi{$\vcenter{\hbox{\small$\bullet$}}$} % custom bullet points
\newenvironment{highlights}{
    \begin{itemize}[
        topsep=0.10 cm,
        parsep=0.10 cm,
        partopsep=0pt,
        itemsep=0pt,
        leftmargin=0 cm + 10pt
    ]
}{
    \end{itemize}
} % new environment for highlights


\newenvironment{highlightsforbulletentries}{
    \begin{itemize}[
        topsep=0.10 cm,
        parsep=0.10 cm,
        partopsep=0pt,
        itemsep=0pt,
        leftmargin=10pt
    ]
}{
    \end{itemize}
} % new environment for highlights for bullet entries

\newenvironment{onecolentry}{
    \begin{adjustwidth}{
        0 cm + 0.00001 cm
    }{
        0 cm + 0.00001 cm
    }
}{
    \end{adjustwidth}
} % new environment for one column entries

\newenvironment{twocolentry}[2][]{
    \onecolentry
    \def\secondColumn{#2}
    \setcolumnwidth{\fill, 4.5 cm}
    \begin{paracol}{2}
}{
    \switchcolumn \raggedleft \secondColumn
    \end{paracol}
    \endonecolentry
} % new environment for two column entries

\newenvironment{threecolentry}[3][]{
    \onecolentry
    \def\thirdColumn{#3}
    \setcolumnwidth{, \fill, 4.5 cm}
    \begin{paracol}{3}
    {\raggedright #2} \switchcolumn
}{
    \switchcolumn \raggedleft \thirdColumn
    \end{paracol}
    \endonecolentry
} % new environment for three column entries

\newenvironment{header}{
    \setlength{\topsep}{0pt}\par\kern\topsep\centering\linespread{1.5}
}{
    \par\kern\topsep
} % new environment for the header

\newcommand{\placelastupdatedtext}{% \placetextbox{<horizontal pos>}{<vertical pos>}{<stuff>}
  \AddToShipoutPictureFG*{% Add <stuff> to current page foreground
    \put(
        \LenToUnit{\paperwidth-2 cm-0 cm+0.05cm},
        \LenToUnit{\paperheight-1.0 cm}
    ){\vtop{{\null}\makebox[0pt][c]{
        \small\color{gray}\textit{Last updated in September 2024}\hspace{\widthof{Last updated in September 2024}}
    }}}%
  }%
}%

% save the original href command in a new command:
\let\hrefWithoutArrow\href

% define space variables
\newlength{\itemSeparate}
\setlength{\itemSeparate}{0.2 cm}

\newlength{\blockSeparate}
\setlength{\blockSeparate}{0.4 cm}
% new command for external links:

% importar comandos y config
\newcommand{\formatTasks}[1]{%
  \begin{highlights}
    % 1. Cambiar todas las "," por "◊" (símbolo poco usado)
    \StrSubstitute{#1}{,}{◊}[\step1]%
    % 2. Cambiar todos los "|" por ","  
    \StrSubstitute{\step1}{|}{,}[\step2]%
    % 3. Separar items por "," y procesarlos
    \renewcommand*{\do}[1]{%
      % 4. En cada item, cambiar "◊" de vuelta a ","
      \StrSubstitute{##1}{◊}{,}[\cleanitem]%
      \item \cleanitem
    }%
    \expandafter\docsvlist\expandafter{\step2}%
  \end{highlights}
}
% Configurar Variables Personales
\newcommand{\name}{Rafael Carvacho}
\newcommand{\email}{r.carvacho.plaza@gmail.com}
\newcommand{\phone}{+56 9 6285 1570}
\newcommand{\address}{Santiago de Chile}
\newcommand{\linkedIn}{https://www.linkedin.com/in/rafael-carvacho/}
\newcommand{\portfolio}{https://unnunoctio-dev.vercel.app/}
\newcommand{\github}{https://github.com/Unnunoctio}



\begin{document}
    \newcommand{\AND}{\unskip
        \cleaders\copy\ANDbox\hskip\wd\ANDbox
        \ignorespaces
    }
    \newsavebox\ANDbox
    \sbox\ANDbox{$|$}

    %-----------------------------------------------
    \begin{header}
        \fontsize{32 pt}{32 pt}\selectfont \name
        
        %\fontsize{14 pt}{14 pt}\selectfont SOFTWARE DEVELOPER

        \vspace{5 pt}

        \normalsize
        \mbox{\address}%
        \kern 5.0 pt%
        \AND%
        \kern 5.0 pt%
        \mbox{\hrefWithoutArrow{mailto:\email}{\email}}%
        \kern 5.0 pt%
        \AND%
        \kern 5.0 pt%
        \mbox{\hrefWithoutArrow{tel:\phone}{\phone}}%
        \kern 5.0 pt%
        \AND%
        \kern 5.0 pt%
        \mbox{\hrefWithoutArrow{\portfolio}{Portafolio}}%
        \kern 5.0 pt%
        \AND%
        \kern 5.0 pt%
        \mbox{\hrefWithoutArrow{\linkedIn}{LinkedIn}}%
        \kern 5.0 pt%
        \AND%
        \kern 5.0 pt%
        \mbox{\hrefWithoutArrow{\github}{Github}}%
    \end{header}

    \vspace{5 pt - 0.3 cm}

    %-----------------------------------------------
    \section{Objetivo}
        \DTLloaddb{obj}{data/objective.csv}
        \DTLforeach{obj}{\objective=objective}{
            \objective
        }
    %-----------------------------------------------
    \section{Experiencia}
        \newcounter{expCounter}
        \DTLloaddb{exp}{data/experience.csv}
        \DTLforeach{exp}{\company=company,\position=position,\period=period,\tasks=tasks}{
            \stepcounter{expCounter}
            \begin{twocolentry}{\textit{\period}}
                \textbf{\company}
            \end{twocolentry}
            \position
            \vspace{\itemSeparate}
            \begin{onecolentry}
                \formatTasks{\tasks}
            \end{onecolentry}
            % Solo agregar espacio si no es el último
            \ifnum\value{expCounter}<\DTLrowcount{exp}
                \vspace{\blockSeparate}
            \fi
        }
    %-----------------------------------------------
    \section{Educación}
        \newcounter{eduCounter}
        \DTLloaddb{edu}{data/education.csv}
        \DTLforeach{edu}{\institution=institution,\title=title,\period=period,\topics=topics}{
            \stepcounter{eduCounter}
            \begin{twocolentry}{\textit{\period}}
                \textbf{\institution}
            \end{twocolentry}
            \title
            % Solo mostrar topics si no está vacío
            \DTLifeq{\topics}{}{}{
                \vspace{\itemSeparate}
                \begin{onecolentry}
                    \formatTasks{\topics}
                \end{onecolentry}
            }
            % Solo agregar espacio si no es el último
            \ifnum\value{eduCounter}<\DTLrowcount{edu}
                \vspace{\blockSeparate}
            \fi
        }
    %-----------------------------------------------
    \newpage
    \section{Certificaciones}
        \newcounter{certCounter}
        \DTLloaddb{cert}{data/certification.csv}
        \DTLforeach{cert}{\institution=institution,\title=title,\period=period,\topics=topics}{
            \stepcounter{certCounter}
            \begin{twocolentry}{\textit{\period}}
                \textbf{\institution}
            \end{twocolentry}
            \title
            % Solo mostrar topics si no está vacío
            \DTLifeq{\topics}{}{}{
                \vspace{\itemSeparate}
                \begin{onecolentry}
                    \formatTasks{\topics}
                \end{onecolentry}
            }
            % Solo agregar espacio si no es el último
            \ifnum\value{certCounter}<\DTLrowcount{cert}
                \vspace{\blockSeparate}
            \fi
        }
    %-----------------------------------------------
    %\section{Publications}
    %    \begin{samepage}
    %        \begin{twocolentry}{ Jan 2004 }
    %            \textbf{3D Finite Element Analysis of No-Insulation Coils}
    %        \end{twocolentry}
    %        \vspace{0.10 cm}
    %        \begin{onecolentry}
    %            \mbox{Frodo Baggins}, \mbox{\textbf{\textit{John Doe}}}, \mbox{Samwise Gamgee}
    %            \vspace{0.10 cm}
    %            \href{https://doi.org/10.1109/TASC.2023.3340648}{10.1109/TASC.2023.3340648}
    %        \end{onecolentry}
    %    \end{samepage}
    %-----------------------------------------------
    
    \section{Proyectos}
        \textbf{Rincón del Curao}
        \hspace{288pt}
        \href{https://dev.rincondelcurao.cl/}{https://dev.rincondelcurao.cl}
        \\Repositorio: \href{https://github.com/Unnunoctio/Rincon-del-Curao}{github.com/Unnunoctio/Rincon-del-Curao}
        
        \vspace{\itemSeparate}
        \begin{onecolentry}
            \begin{highlights}
                \item Website que recopila los precios de diversas cervezas, destilados y vinos vendidos en Chile. Los datos de los productos se obtienen desde Drinks API, donde los precios se actualizan diariamente desde las a las 9 am hasta las 5 pm hora de Chile.
                \item \textbf{Tecnologías Usadas:} TypeScript, TailwindCSS, NodeJS, Bun, React, NextJS, MongoDB, GraphQL
            \end{highlights}
        \end{onecolentry}

        \vspace{\blockSeparate}
        
        \textbf{Drinks API}
        \hspace{299pt}
        \href{https://drinks-api-web.vercel.app}{https://drinks-api-web.vercel.app}
        \\Repositorio: \href{https://github.com/Unnunoctio/Drinks-API}{github.com/Unnunoctio/Drinks-API}

        \vspace{\itemSeparate}
        \begin{onecolentry}
            \begin{highlights}
                \item API REST que maneja diferentes tipos de bebidas alcohólicas, especificando cervezas, licores y vinos. Esta API permite listar y filtrar las diferentes bebidas.
                \item \textbf{Tecnologías Usadas:} JavaScript, Bun, Hono, MongoDB, API REST
            \end{highlights}
        \end{onecolentry}

        \vspace{\blockSeparate}

        \textbf{Github Calendar JSON}
        \hspace{150pt}
        \href{https://www.npmjs.com/package/github-calendar-json}{https://www.npmjs.com/package/github-calendar-json}
        \\Repositorio: \href{https://github.com/Unnunoctio/github-calendar-json}{github.com/Unnunoctio/github-calendar-json}

        \vspace{\itemSeparate}
        \begin{onecolentry}
            \begin{highlights}
                \item Paquete de NPM que transforma el calendario de contribuciones de Github a formato JSON.
                \item \textbf{Tecnologías Usadas:} JavaScript, NodeJS, NPM
            \end{highlights}
        \end{onecolentry}

    %-----------------------------------------------
    
    \section{Tecnologías}
        \begin{onecolentry}
            \textbf{Lenguajes:} HTML, CSS, JavaScript, TypeScript, Python, Java, Julia, C, C++
        \end{onecolentry}

        \vspace{\itemSeparate}

        \begin{onecolentry}
            \textbf{Entornos de Ejecución:} NodeJS, Bun
        \end{onecolentry}

        \vspace{\itemSeparate}

        \begin{onecolentry}
            \textbf{Frameworks:} React, Svelte, NextJS, Astro
        \end{onecolentry}

        \vspace{\itemSeparate}

        \begin{onecolentry}
            \textbf{Bases de Datos:} PostgreSQL, Oracle SQL, MySQL, MongoDB
        \end{onecolentry}

        \vspace{\itemSeparate}

        \begin{onecolentry}
            \textbf{Arquitecturas de APIs:} REST, GraphQL
        \end{onecolentry}

        \vspace{\itemSeparate}

        \begin{onecolentry}
            \textbf{Tecnologías:} Git, GitHub, Docker, VS-Code, Postman, NPM
        \end{onecolentry}

        \vspace{\itemSeparate}

        \begin{onecolentry}
            \textbf{Tecnologías de AWS:} EC2, ECR, ECS, Fargate
        \end{onecolentry}
    
    %-----------------------------------------------
    
\end{document}